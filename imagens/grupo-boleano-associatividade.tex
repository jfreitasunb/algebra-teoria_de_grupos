\documentclass{standalone}

\usepackage{tikz}
\usetikzlibrary{shapes,backgrounds}
\begin{document}
\pagestyle{empty}
% Suppose we have three circles or ellipses or whatever. Let us define
% commands for their paths since we will need them repeatedly in the
% following:

\def\firstcircle{(0,0) circle (1.5cm)}
\def\secondcircle{(45:2cm) circle (1.5cm)}
\def\thirdcircle{(0:2cm) circle (1.5cm)}

% Now we can draw the sets:
\begin{tikzpicture}
    
    % Now we want to highlight the intersection of the first and the
    % second circle:

    \begin{scope}[even odd rule]
      \clip \firstcircle (-3,-3) rectangle (3,3);
      \clip \thirdcircle (-3,-3) rectangle (3,3);
      \fill[blue!45] \secondcircle;
    \end{scope}

    \begin{scope}[even odd rule]
      \clip \secondcircle (-3,-3) rectangle (3,3);
      \clip \thirdcircle (-3,-3) rectangle (3,3);
      \fill[blue!45] \firstcircle;
    \end{scope}

    \begin{scope}[even odd rule]
      \clip \firstcircle (-3,-3) rectangle (4,4);
      \clip \secondcircle (-3,-3) rectangle (4,4);
      \fill[blue!45] \thirdcircle;
    \end{scope}

    \draw \firstcircle node[below] {$A$};
    \draw \secondcircle node [above] {$B$};
    \draw \thirdcircle node [below] {$C$};
    % Next, we want the highlight the intersection of all three circles:

    \begin{scope}
      \clip \firstcircle;
      \clip \secondcircle;
      \fill[blue!45] \thirdcircle;
    \end{scope}
    
\end{tikzpicture}
\end{document}