%!TEX program = xelatex 
%!TEX encoding = ISO-8859-1
\documentclass[12pt]{article}
\usepackage{amssymb}
\usepackage{amsmath,amsfonts,amsthm,amstext}
\usepackage[brazil]{babel}
%\usepackage[latin1]{inputenc}
\usepackage{graphicx}
\graphicspath{{/home/jfreitas/Dropbox/imagens-latex/}{/Users/jfreitas/Dropbox/imagens-latex/}}
\usepackage{enumitem}
\usepackage{multicol}
\usepackage[all]{xy}

%\usepackage{url}


\newcommand{\real}{\mathbb{R}}

%\DeclareGraphicsRule{jpg}{*[}{}{`jpeg2eps #1.jpg}
%\input{seteps}
%\input{setbmp-dvips}

\setlength{\topmargin}{-1.0in}
\setlength{\oddsidemargin}{0in}
\setlength{\textheight}{10.1in}
\setlength{\textwidth}{6.5in}
\setlength{\baselineskip}{12mm}

\begin{document}
\pagestyle{empty}

\begin{figure}[h]
    \begin{minipage}[c]{2cm}
    \includegraphics[width=2cm]{ufv.pdf}
    \end{minipage}%
    \hspace{0pt}
    \begin{minipage}[c]{4in}
    {Universidade Federal de Vi\c{c}osa} \\
    {Departamento de Matem{\'a}tica}
    \end{minipage}
\end{figure}
\vspace{-0.2cm}
\hrule

\begin{center}
{\large\bf Plano de Ensino} \\
{\large\bf \'Algebra - Ver\~ao 2015}\\
Prof. Jos{\'e} Ant{\^o}nio O. Freitas
\end{center}
\hrule
\vspace{0.25cm}
\noindent {\bf{PROGRAMA:}} 6 semanas.
\begin{enumerate}[label=({\arabic*})]
\item Grupos, Subgrupos, Teorema de Lagrange, Grupos c{\'\i}clicos, Subgrupos Normais, Grupos Quociente, Teorema de Isomorfismos de grupos, Teorema de Cayley, Teorema de Sylow.

\item Grupos livres, Grupos abelianos finitamente gerados, Grupos Sol\'uveis, Grupos Nilpotentes.

\item Introdu\c{c}\~ao a teoria dos an\'eis: An\'eis e homomorfismos; algumas classes especiais de an\'eis; dom{\'\i}nios de ideais principais; dom{\'\i}nios euclidianos; an\'eis de polin\^omios; fatora\c{c}\~ao \'unica em an\'eis de polin\^omios. (Esta parte depender\'a do tempo dispon{\'\i}vel.)

\end{enumerate}

\noindent {\bf{BIBLIOGRAFIA:}}
\begin{itemize}
\item {\it Advanced Modern Algebra}, Rotman, J. J.; Prentice Hall, 2003.
\item {\it Algebra}, Lang, S.; Boston: Addison-Wesley, 1984.
\item {\it Elementos de \'Algebra}, Garcia, A., Lequain, Y.; IMPA, 2010.
\item {\it Introdu\c{c}\~ao \`a \'Algebra}, Gon\c{c}alves, A.; Projeto Euclides, Impa, 2006.
\end{itemize}

\noindent {\bf{SISTEMA DE AVALIA\c{C}\~{A}O:}} Ser\~ao realizadas duas provas, as quais s\~{a}o atribu\'{\i}das as notas $P_1$ e $P_2$, nas datas detalhadas a seguir.

\begin{center}
    \begin{tabular}{c|c|c}
        \hline\hline
        \hspace{1cm}{\bf Prova}\hspace{1cm} & \hspace{3cm}{\bf Data}\hspace{3cm} & \hspace{1.7cm}{\bf Hor\'{a}rio}\hspace{1.7cm} \\
        \hline\hline
        $P_1$ & 26/01/15 (segunda-feira) \phantom{x} & 10:00 - 12:00 \\
        \hline
        $P_2$ & 11/02/15 (quarta-feira) \phantom{x} & 10:00 - 12:00 \\
        \hline\hline
        %$PF$ &  11/07/11 (segunda-fei) \phantom{x} & 8:00 - 9:50 \\
        %\hline\hline
    \end{tabular}
\end{center}

{\bf \noindent Nota Final:} A partir das notas das provas mencionadas neste texto, a nota final ($NF$) de cada estudante \'{e} dada
por
\vspace{-0.15cm}
\[
NF = \frac{ P_1 + P_2}{2}
\]
e ser\'{a} considerado aprovado o estudante que obtiver $NF \geq 6,00$.

% {\bf \noindent Men\c{c}\~{a}o Final:} ser\'{a} obtida da $NF$ de
% acordo com as normas da UnB.
% \begin{center}
%     \begin{tabular}{c|c}
%         \hline\hline
%         \hspace{1cm}{Nota}\hspace{1cm} & \hspace{0.25cm}{Men\c{c}\~{a}o}\hspace{0.25cm}\\
%         \hline\hline
%         9,00 a 10,0 & SS \\
%         \hline
%         7,00 a 8,99 & MS \\
%         \hline
%         5,00 a 6,99 & MM \\
%         \hline
%         3,00 a 4,99 & MI \\
%         \hline
%         0,00 a 2,99  & II \\
%         \hline\hline
%     \end{tabular}
% \end{center}
% Receber{\'a} a men{\c c}{\~a}o {\bf SR} quem estiver reprovado por faltar mais de 25\%
% das aulas.

% \noindent {\bf{P\'{A}GINA DA TURMA:}} Todos estudantes devem
% obrigatoriamente se cadastrar (de prefer\^encia, com nome e matr\'icula) no MOODLE no endere\c{c}o \url{www.ead.unb.br}. Em seguida, devem se inscrever na disciplina
% \begin{center}
% ``Geometria Anal{\'\i}tica - Turma B/2014" usando o c\'odigo: ``GA2014-2-turmaB".
% \end{center}

% \begin{itemize}
% \item Toda a comunica\c{c}\~{a}o oficial do curso, inclusive a divulga\c{c}\~{a}o de
% notas e gabaritos, se dar\'{a} atrav\'{e}s do {\em F\'{o}rum de Not\'{\i}cias} do
% MOODLE.\vspace{-0.20cm}
% \item No {\em F\'{o}rum de Debates} do MOODLE poder\~{a}o ser
% postadas d\'{u}vidas que ser\~{a}o respondidas on-line pelos seus
% colegas ou pelo monitor dessa turma.
% \end{itemize}

\noindent {\bf{OBSERVA\c{C}\~{O}ES IMPORTANTES:}}

\begin{enumerate}[label=({\arabic*})]
\item As provas ser\~{a}o individuais e sem qualquer tipo de
aux\'{\i}lio (calculadora, livros etc.), sendo vedado o empr\'{e}stimo de
qualquer material entre os alunos durante as avalia\c{c}\~{o}es. As
tentativas de fraude ser\~{a}o reprimidas com m\'{a}ximo rigor.
\vspace{-0.25cm}

\item \'{E} vedado o uso de telefones celulares e quaisquer dispositivos eletr\^{o}nicos pessoais durante a realiza\c{c}\~{a}o das atividades do curso em sala de aula. \vspace{-0.25cm}


\item A aus\^{e}ncia acarretar\'{a} nota zero em qualquer uma das
avalia\c{c}\~{o}es. \vspace{-0.25cm}

\item A crit\'{e}rio do professor, as datas das provas poder\~{a}o
ser alteradas. \vspace{-0.25cm}

\item Haver{\'a} avalia{\c c}{\~a}o quanto {\`a} clareza, apresenta{\c
c}{\~a}o e formaliza{\c c}{\~a}o na  resolu{\c c}{\~a}o das quest{\~o}es de
cada prova. A nota do aluno poder{\'a} ser alterada em raz{\~a}o da
inobserv{\^a}ncia desses par{\^a}metros.
\end{enumerate}


\vfill
\hrule
\end{document}
