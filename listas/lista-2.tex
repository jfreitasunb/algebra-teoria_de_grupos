%!TEX program = xelatex 
% !TEX encoding = ISO-8859-1
\documentclass[12pt]{article}

\usepackage{amssymb}
\usepackage{amsmath,amsfonts,amsthm,amstext}
\usepackage[brazil]{babel}
%\usepackage[latin1]{inputenc}
\usepackage{graphicx}
\graphicspath{{/home/jfreitas/Dropbox/imagens-latex/}{/Users/jfreitas/Dropbox/imagens-latex/}}
\usepackage{enumitem}
\usepackage{multicol}
\usepackage[all]{xy}

\setlength{\topmargin}{-1.0in}
\setlength{\oddsidemargin}{0in}
\setlength{\textheight}{10.1in}
\setlength{\textwidth}{6.5in}
\setlength{\baselineskip}{12mm}

\newcounter{exercicios}
\setcounter{exercicios}{0}
\newcommand{\questao}{
\addtocounter{exercicios}{1}
\noindent{\bf Exerc{\'\i}cio \arabic{exercicios}: }}

\newcommand{\aut}{{\rm Aut\,}}
\newcommand{\equi}{\Leftrightarrow}
\newcommand{\bic}{\leftrightarrow}
\newcommand{\cond}{\rightarrow}
\newcommand{\impl}{\Rightarrow}
\newcommand{\nao}{\sim}
\newcommand{\sub}{\subseteq}
\newcommand{\e}{\ \wedge\ }
\newcommand{\ou}{\ \vee\ }
\newcommand{\vaz}{\emptyset}
\newcommand{\ger}[1]{\langle{#1}\rangle}

\newcommand{\R}{\mathbb{R}}
\newcommand{\cmp}{\mathbb{C}}
\newcommand{\vesp}{\vspace{0.2cm}}
\newcommand{\Z}{\mathbb{Z}}
\newcommand{\N}{\mathbb{N}}
\newcommand{\Q}{\mathbb{Q}}
\newtheorem{defin}{Defini{\c c}{\~a}o}

\newcommand{\compcent}[1]{\vcenter{\hbox{$#1\circ$}}}
\newcommand{\comp}{\mathbin{\mathchoice
{\compcent\scriptstyle}{\compcent\scriptstyle}
{\compcent\scriptscriptstyle}{\compcent\scriptscriptstyle}}}

\begin{document}
\pagestyle{empty}

\begin{figure}[h]
        \begin{minipage}[c]{2cm}
        \includegraphics[width=2cm]{ufv.pdf}
        \end{minipage}%
        \hspace{0pt}
        \begin{minipage}[c]{4in}
          {Universidade Federal de Vi\c{c}osa} \\
          {Departamento de Matem{\'a}tica}
\end{minipage}
\end{figure}
\vspace{-0.2cm}\hrule

\begin{center}
{\Large\bf {\'A}lgebra - Curso de Ver\~ao - UFV} \\ \vspace{9pt} {\large\bf
  $2^{\underline{a}}$ Lista de Exerc{\'\i}cios -- 2015}\\ \vspace{9pt} Prof. Jos{\'e} Ant{\^o}nio O. Freitas
\end{center}
\hrule

\vspace{.6cm}

\questao Seja $Q_8 = \{\pm 1, \pm i, \pm j, \pm k\}$, onde $i^2 = j^2 = k^2 = -1$ e $ij = k$, $jk = i$, $ki = j$, $ji = -k$, $kj = -i$, $ik = -j$. Mostre que $Q_8$ \'e um grupo e que $Q_8 \cong \mathcal{Q}$, onde
\[
    \mathcal{Q} = \{I, A, A^2, A^3, B, BA, BA^2, BA^3\},
\]
$I$ \'e a matriz identidade e
    \[
      A = \begin{bmatrix}
        0 & 1\\-1 & 0
      \end{bmatrix}, \quad B = \begin{bmatrix}
        0 & i\\i & 0
      \end{bmatrix}
    \]
com $i \in \cmp$, $i^2 = -1$.
\vesp

\questao Calcule todos os subgrupos grupos dados. Quais subgrupos s\~ao normais?
\begin{enumerate}[label=({\alph*})]
  \item $S_3$
  \item $D_8$
  \item $D_6$
  \item $Q_8$
\end{enumerate}

\vesp

\questao Seja $G$ um grupo. Defina $G' =  \ger{\{xyx^{-1}y^{-1} \mid x, y \in G\}}$. Mostre que:
\begin{enumerate}[label=({\alph*})]
  \item $G'$ \'e um subgrupo normal de $G$.
  \item $G/G'$ \'e abeliano.
  \item $G'$ \'e o menor subgrupo normal de $G$ com esta propriedade, isto \'e, se $H \unlhd G$ \'e tal que $G/H$ \'e abeliano, ent\~ao $G'\sub H$.
\end{enumerate}
O subgrupo $G'$ \'e chamado de \textbf{subgrupo de comutadores}.

\vesp

\questao Seja $G$ um grupo tal que $\{1\}$ e $G$ s\~ao seus \'unicos subgrupos. Mostre que a ordem de $G$ \'e um n\'umero primo.

\vesp

\questao Sejam $G$ e $H$ grupos e $\phi : G \to H$ um homomorfismo. Mostre que se $|x| < \infty$, ent\~ao $|\phi(x)|$ divide $|x|$.

\vesp

\questao Seja $\phi : G \to H$ um homomorfismo de grupos. Se $\phi$ \'e injetivo, mostre que $|\phi(x)| = |x|$ para todo $x \in G$.

\vesp

\questao Seja $\phi : G \to H$ um isomorfimo de grupos. Mostre que:
\begin{enumerate}[label=({\alph*})]
  \item Se $a \in G$ tem ordem infinita, ent\~ao $\phi(a)$ tamb\'em tem ordem infinita.
  \item Se $a \in G$ tem ordem $n$, ent\~ao $\phi(a)$ tamb\'em tem ordem $n$.
  \item Conclua que se $G$ tem um elemento de ordem $n$ e $H$ n\~ao possui elemento com essa ordem, ent\~ao $G \ncong H$.
\end{enumerate}
 

\vesp

\questao Prove que um grupo $G$ \'e abeliano, se e somente se, a fun\c{c}\~ao $f : G \to G$ dada por $f(a) = a^{-1}$ \'e um homomorfismo.

\vesp

\questao Seja $G$ um grupo e $H$ um subgrupo de $G$. Mostre que se $[G:H] = 2$, ent\~ao $H \unlhd G$.

\vesp

\questao Sejam $G$ e $H$ grupos e $\phi : G \to H$ um homomorfismo. Mostre que $\ker\phi \unlhd G$.

\vesp

\questao \'E verdade que se $K \unlhd H \unlhd G$, ent\~ao $H \unlhd G$?

\vesp

\questao Seja $G$ um grupo. Um isomorfismo $\phi : G \to G$ \'e chamado de um \textbf{automorfismo} de $G$. Seja $\aut G = \{\phi : G \to G \mid \phi \mbox{ \'e um automorfismo de } G\}$. Mostre que $\aut G$ \'e um grupo com a composi\c{c}\~ao de fun\c{c}\~oes.

\vesp

\questao Sejam $G$ um grupo e $\mathcal{I}_a : G \to G$, para $a \in G$ fixado, definido por $\mathcal{I}_a(x) = a^{-1}xa$.
\begin{enumerate}[label=({\alph*})]
  \item Mostre que $\mathcal{I}_a$ \'e um isomorfismo.
  \item Seja $\mathcal{I}(G) = \{\mathcal{I}_a \mid a \in G\} \sub \aut(G)$. Mostre que $\mathcal{I}(G)$ \'e um subgrupo normal de $\aut(G)$.
\end{enumerate}

\vesp

\questao Considere a fun\c{c}\~ao
\begin{align*}
  \mathcal{I} : (G, \cdot) &\to (\mathcal{I}(G),\comp)\\
  a &\mapsto \mathcal{I}_a.
\end{align*}
Por defini\c{c}\~ao, $\mathcal{I}$ \'e uma fun\c{c}\~ao sobrejetora.
\begin{enumerate}[label=({\alph*})]
  \item Mostre que $\mathcal{I}$ \'e um homomorfismo de grupos.
  \item Mostre que $\ker\mathcal{I} = Z(G)$ e que $\mathcal{I}(G) \cong G /Z(G)$.
  \item Mostre que se $G$ n\~ao \'e abeliano, ent\~ao $\mathcal{I}(G)$ n\~ao \'e c{\'\i}clico.
\end{enumerate}

\questao Seja $G$ um grupo finito e sejam $K < H < G$. Mostre que
  \[
    [G:K] = [G:H][H:K].
  \]

\vesp

\questao Sejam $G$ um grupo e $a$, $b \in G$. Mostre que $(a^{-1}ba)^n = a^{-1}b^na$ para todo $n \in \Z$.

\vesp

\questao Seja $G$ um grupo. Mostre que se $H \unlhd G$ e $K \le G$, ent\~ao
\[
  \dfrac{K}{H \cap K} \cong \dfrac{HK}{H}.
\]

\vesp

\questao Seja $G$ um grupo. Mostre que se $K \le H \le G$ com $K \unlhd G$ e $H \unlhd G$, ent\~ao
\[
  \dfrac{G/K}{H/K} \cong \dfrac{G}{H}.
\]

\vesp

\questao Mostre que todo grupo $G$ tal que $|G| < 6$ \'e abeliano.

\vesp

\questao Mostre que se $G$ \'e um grupo de ordem 6, ent\~ao ou $G$ \'e c{\'\i}clico ou $G \cong S_3$.

\vesp

\questao Seja $G = \{f : \R \to \R \mid f(x) = ax + b, a \ne 0\}$. Prove que $G$ \'e um grupo com a composi\c{c}\~ao de fun\c{c}\~oes que \'e isomorfo ao subgrupo de $GL_2(\R)$ formado pelas matrizes do tipo
\[
  \begin{bmatrix}
    a & b\\
    0 & 1
  \end{bmatrix}.
\]

\vesp

\questao Seja $p$ um n\'umero primo e $G$ um $p$-grupo de ordem $p^3$. Prove que se $G$ n\~ao \'e abeliano, ent\~ao $|Z(G)| = p$.

\vesp

\questao Mostre que todo grupo quociente de um grupo c{\'\i}clico \'e c{\'\i}clico.

\vesp

\questao Seja $G$ um grupo contendo apenas duas classes de conjuga\c{c}\~ao. Mostre que $|G| = 2$, isto \'e, $G$ \'e um grupo c{\'\i}clico de ordem 2.

\vesp

\questao Seja $G$ um grupo tal que $|G| = 2p$, onde $p$ \'e um n\'umero primo. Mostre que existe $H \unlhd G$ tal que $|H| = p$.

\vesp

\questao Seja $G$ um grupo tal que $|G| = pq$, onde $p$ e $q$ s\~ao primos. Mostre que se $G$ \'e abeliano e $p \ne q$, ent\~ao $G$ \'e c{\'\i}clico.
\end{document}