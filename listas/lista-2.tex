%!TEX program = xelatex 
% !TEX encoding = ISO-8859-1
\documentclass[12pt]{article}

\usepackage{amssymb}
\usepackage{amsmath,amsfonts,amsthm,amstext}
\usepackage[brazil]{babel}
%\usepackage[latin1]{inputenc}
\usepackage{graphicx}
\graphicspath{{/home/jfreitas/Dropbox/imagens-latex/}{/Users/jfreitas/Dropbox/imagens-latex/}}
\usepackage{enumitem}
\usepackage{multicol}
\usepackage[all]{xy}

\setlength{\topmargin}{-1.0in}
\setlength{\oddsidemargin}{0in}
\setlength{\textheight}{10.1in}
\setlength{\textwidth}{6.5in}
\setlength{\baselineskip}{12mm}

\newcounter{exercicios}
\setcounter{exercicios}{0}
\newcommand{\questao}{
\addtocounter{exercicios}{1}
\noindent{\bf Exerc{\'\i}cio \arabic{exercicios}: }}

\newcommand{\equi}{\Leftrightarrow}
\newcommand{\bic}{\leftrightarrow}
\newcommand{\cond}{\rightarrow}
\newcommand{\impl}{\Rightarrow}
\newcommand{\nao}{\sim}
\newcommand{\sub}{\subseteq}
\newcommand{\e}{\ \wedge\ }
\newcommand{\ou}{\ \vee\ }
\newcommand{\vaz}{\emptyset}
\newcommand{\ger}[1]{\langle{#1}\rangle}

\newcommand{\R}{\mathbb{R}}
\newcommand{\cmp}{\mathbb{C}}
\newcommand{\vesp}{\vspace{0.2cm}}
\newcommand{\Z}{\mathbb{Z}}
\newcommand{\N}{\mathbb{N}}
\newcommand{\Q}{\mathbb{Q}}
\newtheorem{defin}{Defini{\c c}{\~a}o}

\newcommand{\compcent}[1]{\vcenter{\hbox{$#1\circ$}}}
\newcommand{\comp}{\mathbin{\mathchoice
{\compcent\scriptstyle}{\compcent\scriptstyle}
{\compcent\scriptscriptstyle}{\compcent\scriptscriptstyle}}}

\begin{document}
\pagestyle{empty}

\begin{figure}[h]
        \begin{minipage}[c]{2cm}
        \includegraphics[width=2cm]{ufv.pdf}
        \end{minipage}%
        \hspace{0pt}
        \begin{minipage}[c]{4in}
          {Universidade Federal de Vi\c{c}osa} \\
          {Departamento de Matem{\'a}tica}
\end{minipage}
\end{figure}
\vspace{-0.2cm}\hrule

\begin{center}
{\Large\bf {\'A}lgebra - Curso de Ver\~ao - UFV} \\ \vspace{9pt} {\large\bf
  $2^{\underline{a}}$ Lista de Exerc{\'\i}cios -- 2015}\\ \vspace{9pt} Prof. Jos{\'e} Ant{\^o}nio O. Freitas
\end{center}
\hrule

\vspace{.6cm}

\questao Seja $G$ um grupo tal que $x^1 = 1$ para todo $x \in G$. Mostre que $G$ é abeliano.

\vesp

\questao Seja $G$ um grupo. Defina $G' =  \ger{\{xyx^{-1}y^{-1} \mid x, y \in G\}}$. Mostre que
\begin{enumerate}[label=({\alph*})]
  \item $G'$ é um subgrupo normal de $G$.
  \item $G/G'$ \'e abeliano.
  \item $G'$ \'e o menor subgrupo normal de $G$ com esta propriedade, isto \'e, se $H \unlhd G$ \'e tal que $G/H$ \'e abeliano, ent\~ao $G'\sub H$.
\end{enumerate}
O subgrupo $G'$ é chamado de \textbf{subgrupo de comutadores}.

\vesp

\questao Seja $G$ um grupo finito e sejam $K < H < G$. Mostre que
  \[
    [G:K] = [G:H][H:K].
  \]

\vesp

\questao Sejam $G$ um grupo e $a$, $b \in G$. Mostre que $(a^{-1}ba)^n = a^{-1}b^na$ para todo $n \in \Z$.

\vesp

\questao Seja $G$ um grupo. Mostre que se $H \unlhd G$ e $K \le G$, então
\[
  \dfrac{K}{H \cap K} \cong \dfrac{HK}{H}.
\]

\vesp

\questao Sem $G$ um grupo. Mostre que se $K \le H \le G$ com $K \unlhd G$ e $H \unlhd G$, então
\[
  \dfrac{G/K}{H/K} \cong \dfrac{G}{H}.
\]

\end{document}