%!TEX program = xelatex 
%!TEX encoding = ISO-8859-1
\documentclass[12pt]{article}

\usepackage{amssymb}
\usepackage{amsmath,amsfonts,amsthm,amstext}
\usepackage[brazil]{babel}
%\usepackage[latin1]{inputenc}
\usepackage{graphicx}
\graphicspath{{/home/jfreitas/Dropbox/imagens-latex/}{/Users/jfreitas/Dropbox/imagens-latex/}}
\usepackage{enumitem}
\usepackage{multicol}
\usepackage[all]{xy}

\setlength{\topmargin}{-1.0in}
\setlength{\oddsidemargin}{0in}
\setlength{\textheight}{10.1in}
\setlength{\textwidth}{6.5in}
\setlength{\baselineskip}{12mm}

\newcounter{exercicios}
\setcounter{exercicios}{0}
\newcommand{\questao}{
\addtocounter{exercicios}{1}
\noindent{\bf Exerc{\'\i}cio \arabic{exercicios}: }}

\newcommand{\aut}{{\rm Aut\,}}
\newcommand{\rnk}{{\rm posto\,}}
\newcommand{\sgn}{{\rm sgn\,}}
\newcommand{\equi}{\Leftrightarrow}
\newcommand{\bic}{\leftrightarrow}
\newcommand{\cond}{\rightarrow}
\newcommand{\impl}{\Rightarrow}
\newcommand{\nao}{\sim}
\newcommand{\sub}{\subseteq}
\newcommand{\e}{\ \wedge\ }
\newcommand{\ou}{\ \vee\ }
\newcommand{\vaz}{\emptyset}
\newcommand{\ger}[1]{\langle{#1}\rangle}

\newcommand{\R}{\mathbb{R}}
\newcommand{\cmp}{\mathbb{C}}
\newcommand{\vesp}{\vspace{0.2cm}}
\newcommand{\Z}{\mathbb{Z}}
\newcommand{\N}{\mathbb{N}}
\newcommand{\Q}{\mathbb{Q}}
\newtheorem{defin}{Defini{\c c}{\~a}o}

\newcommand{\compcent}[1]{\vcenter{\hbox{$#1\circ$}}}
\newcommand{\comp}{\mathbin{\mathchoice
{\compcent\scriptstyle}{\compcent\scriptstyle}
{\compcent\scriptscriptstyle}{\compcent\scriptscriptstyle}}}

\begin{document}
\pagestyle{empty}

\begin{figure}[h]
        \begin{minipage}[c]{2cm}
        \includegraphics[width=2cm]{ufv.pdf}
        \end{minipage}%
        \hspace{0pt}
        \begin{minipage}[c]{4in}
          {Universidade Federal de Vi\c{c}osa} \\
          {Departamento de Matem{\'a}tica}
\end{minipage}
\end{figure}
\vspace{-0.2cm}\hrule

\begin{center}
{\Large\bf {\'A}lgebra - Curso de Ver\~ao - UFV} \\ \vspace{9pt} {\large\bf
  $4^{\underline{a}}$ Lista de Exerc{\'\i}cios -- 2015}\\ \vspace{9pt} Prof. Jos{\'e} Ant{\^o}nio O. Freitas
\end{center}
\hrule

\vspace{.6cm}

\questao Seja $G$ um grupo tal que $|G| = p(p + 2)$, onde $p$ e $p + 2$ s\~ao primos (chamados \textbf{primos g\^emeos}). Mostre que $G$ \'e c{\'\i}clico.

\vesp

\questao Prove que todo grupo de ordem $5\cdot 7\cdot 47$ \'e c{\'\i}clico.

\vesp

\questao Seja $G$ um grupo finito. Mostre que:
\begin{enumerate}[label=({\alph*})]
  \item Se $|G| = 42$, ent\~ao $n_7 = 1$.

  \item Se $|G| = 48$, ent\~ao $G$ necessariamente cont\'em um sugbrupo normal de ordem 8 ou de ordem 16.

  \item Se $|G| = 36$, ent\~ao $G$ cont\'em um subgrupo normal de ordem 9 ou 3.
\end{enumerate}

\vesp

\questao Sejam $G$ um grupo, $|G| = p^mb$, com $p$ n\'umero primo e $p$ n\~ao divide $b$, $K$ um $p$-subgrupo de Sylow de $G$ e $H \unlhd G$ tal que $K \sub H$. Mostre que $K \unlhd H$ se, e somente se, $K \unlhd G$ se, e somente se, $n_p = 1$.

\vesp

\questao Prove que n\~ao existem grupos simples de ordem 28 ou 312.

\vesp

\questao Sejam $G$ um grupo finito tal que $|G| = p_1p_2\cdots p_r$ com $p_1 < p_2 < \cdots < p_r$ e, para cada $i$, $p_i$ \'e primo. Sabendo que grupos deste tipo n\~ao s\~ao simples, mostre que o $p_r$ subgrupo de Sylow de $G$ \'e normal.

\vesp

\questao Sejam $p$ um n\'umero primo e $G$ um grupo n\~ao abeliano de ordem $p^3$. Mostre que $|Z(G)| = p$. Mostre que $Z(G) = G'$  e que $G/Z(G) \cong \Z/p\Z \times \Z/p\Z$.

\vesp

\questao Seja $G$ um grupo de ordem $11^213^2$. Mostre que $G$ \'e um grupo abeliano.

\vesp

\questao Sejam $G$ um $p$-grupo finito, isto \'e, $|G| = p^n$ e $H \le G$. Mostre que:
\begin{enumerate}[label=({\alph*})]
  \item Se $H \ne G$, ent\~ao existe $x \in G$, $x \notin H$ tal que $x^{-1}Hx = H$. [Sugest\~ao: Fa\c{c}a por indu\c{c}\~ao sobre $n$ usando as possibilidades de $Z(G)$ estar ou n\~ao contido em $H$.]

  \item Se $|H| = p^{n - 1}$, ent\~ao $H$ \'e normal em $G$.

  \item Existe uma sequ\^encia de subgrupos $H_0 \le H_1 \le \cdots \le H_n$ tal que $H_i \unlhd  H_{i + 1}$, $i = 0$, \dots, $n - 1$ e $H_{i + 1}/H_i$ \'e c{\'\i}clido de ordem $p$.
\end{enumerate}

\vesp

\questao Sejam $x$, $y$, $z$ e $t$ elementos de um grupo $G$. Mostre que:
\begin{enumerate}[label=({\alph*})]
  \item $[x, y]^{-1} = [y,x]$.
  \item $[x, y]^z = [x^z,y^z]$.
  \item $[x, y]z = z[x^z,y^z]$.
  \item $[x^y, z] = [x,z]^{[x,y]}[x,y,z]$.
\end{enumerate}

\vesp

\questao Um grupo $G$ \'e chamado de \textbf{perfeito} se $G = G'$. Provar que todo grupo que n\~ao \'e sol\'uvel cont\'em um sugbrupo caracter{\'\i}stico $H \ne \{1\}$ que \'e perfeito.

\vesp

\questao Sejam $H$ e $K$ dois subgrupos normais de um grupo $G$. Prove que se ambos $H$ e $K$ s\~ao sol\'uveis, ent\~ao o grupo $HK$ tamb\'em \'e sol\'uvel.

\vesp

\questao Prove que todo grupo de ordem $12$ \'e sol\'uvel.

\vesp

\questao Sejam $p \ne q$ dois n\'umeros primos. Prove que todo grupo de ordem $pq$ \'e sol\'uvel.

\vesp

\questao Seja $G$ um grupo finito nilpotente de ordem $n$. Prove que, para cada divisor $d$ de $n$, $G$ cont\'em um sugrupo de ordem $d$.

\vesp

\questao Seja $H$ um subgrupo de um grupo finito nilpotente $G$. Definimos $N_1 = N_G(H)$ e, indutivamente, $N_i = N_G(N_{i - 1})$. Prove que existe um inteiro positivo $k$ tal que $N_k = G$.

\vesp

\questao Mostre que, se um grupo $G$ \'e tal que $G/Z(G)$ \'e nilpotente, ent\~ao $G$ \'e nilpotente.

\vesp

\questao Sejam $G$ e $H$ grupos abelianos finitamente gerados. Prove que $G \times G \cong H \times H$ se, e somente se, $G \cong H$.

\vesp

\questao Sejam $G$, $H$ e $K$ grupos abelianos finitamente gerados. Prove que $G \times K \cong H \times K$ se, e somente se, $G \cong H$.

\vesp

\questao Prove que vale a rec{\'\i}prova do Teorema de Lagrange para grupos abelianos finitos.

\vesp

\questao Quantos grupos abelianos $G$ existem, a menos de isomorfismo, de ordem:
\begin{enumerate}[label=({\alph*})]
  \item $|G| = 2700$
  \item $|G| = 7^211^213$
  \item $|G| = p^10$, com $p$ primo.
\end{enumerate}

\vesp

\questao Sejam $G$ e $H$ grupos abelianos de ordem $p^n$, com $p$ primo. Mostre que:
\begin{enumerate}[label=({\alph*})]
  \item Se $pG = \{0\}$, ent\~ao $G \cong \Z_p^n$.
  \item $pG \cong pH$ se, e somente se, $G \cong H$.
\end{enumerate}

\vesp

\questao Quantos grupos abelianos $G$ existem, a menos de isomorfismo, satisfazendo:
\begin{enumerate}[label=({\alph*})]
  \item $|G| = 3^5$, $9G = \{0\}$ e $3G \ne \{0\}$
  \item $|G| = 5^4$, $25G = \{0\}$ e $|5G| = 25$
  \item $|G| = 7^511^3$ e $77G = \{0\}$
  \item $|G| = 32$, para todo $x \in G$ tem-se que $|x| \le 8$ e existe $x \in G$ com $|x| = 8$.
\end{enumerate}

\vesp

\questao Quantos grupos abelianos finitamente gerados existem, a menos de isomorfismo, satisfazendo:
\begin{enumerate}[label=({\alph*})]
  \item $|T(G)| = 27$, $\rnk(G) = 3$ e para todo $x \in G$ tem-se que $|x| = 3$ ou $|x| = \infty$.
  \item $|T(G)| = 25$, $\rnk(G) = 4$ e $5G$ \textbf{n\~ao} \'e livre.
\end{enumerate}
\end{document}