%!TEX program = xelatex 
% !TEX encoding = ISO-8859-1
\documentclass[12pt]{article}

\usepackage{amssymb}
\usepackage{amsmath,amsfonts,amsthm,amstext}
\usepackage[brazil]{babel}
%\usepackage[latin1]{inputenc}
\usepackage{graphicx}
\graphicspath{{/home/jfreitas/Dropbox/imagens-latex/}{/Users/jfreitas/Dropbox/imagens-latex/}}
\usepackage{enumitem}
\usepackage{multicol}
\usepackage[all]{xy}

\setlength{\topmargin}{-1.0in}
\setlength{\oddsidemargin}{0in}
\setlength{\textheight}{10.1in}
\setlength{\textwidth}{6.5in}
\setlength{\baselineskip}{12mm}

\newcounter{exercicios}
\setcounter{exercicios}{0}
\newcommand{\questao}{
\addtocounter{exercicios}{1}
\noindent{\bf Exerc{\'\i}cio \arabic{exercicios}: }}

\newcommand{\equi}{\Leftrightarrow}
\newcommand{\bic}{\leftrightarrow}
\newcommand{\cond}{\rightarrow}
\newcommand{\impl}{\Rightarrow}
\newcommand{\nao}{\sim}
\newcommand{\sub}{\subseteq}
\newcommand{\e}{\ \wedge\ }
\newcommand{\ou}{\ \vee\ }
\newcommand{\vaz}{\emptyset}
\newcommand{\ger}[1]{\langle{#1}\rangle}

\newcommand{\R}{\mathbb{R}}
\newcommand{\cmp}{\mathbb{C}}
\newcommand{\vesp}{\vspace{0.2cm}}
\newcommand{\Z}{\mathbb{Z}}
\newcommand{\N}{\mathbb{N}}
\newcommand{\Q}{\mathbb{Q}}
\newtheorem{defin}{Defini{\c c}{\~a}o}

\newcommand{\compcent}[1]{\vcenter{\hbox{$#1\circ$}}}
\newcommand{\comp}{\mathbin{\mathchoice
{\compcent\scriptstyle}{\compcent\scriptstyle}
{\compcent\scriptscriptstyle}{\compcent\scriptscriptstyle}}}

\begin{document}
\pagestyle{empty}

\begin{figure}[h]
        \begin{minipage}[c]{2cm}
        \includegraphics[width=2cm]{ufv.pdf}
        \end{minipage}%
        \hspace{0pt}
        \begin{minipage}[c]{4in}
          {Universidade Federal de Vi\c{c}osa} \\
          {Departamento de Matem{\'a}tica}
\end{minipage}
\end{figure}
\vspace{-0.2cm}\hrule

\begin{center}
{\Large\bf {\'A}lgebra - Curso de Ver\~ao - UFV} \\ \vspace{9pt} {\large\bf
  $1^{\underline{a}}$ Lista de Exerc{\'\i}cios -- 2015}\\ \vspace{9pt} Prof. Jos{\'e} Ant{\^o}nio O. Freitas
\end{center}
\hrule

\vspace{.6cm}

\questao Verifique se o conjunto $\Q_{>0}$ dos n{\'u}meros racionais estritamente positivos com a
 opera{\c c}{\~a}o dada {\'e} ou n{\~a}o um grupo. Justifique sua
resposta.
\begin{multicols}{2}
\begin{enumerate}[label=({\alph*})]
\item $(\Q_{>0},\cdot)$
\item $(\Q_{>0}, +)$
\end{enumerate}
\end{multicols}

\vesp

\questao Seja $\mathcal{Q}$ o conjunto dado por
    \[
      \mathcal{Q} = \{I, A, A^2, A^3, B, BA, BA^2, BA^3\}
    \]
onde $I$ \'e a matriz identidade e
    \[
      A = \begin{bmatrix}
        0 & 1\\-1 & 0
      \end{bmatrix}, \quad B = \begin{bmatrix}
        0 & i\\i & 0
      \end{bmatrix}
    \]
onde $i \in \cmp$, $i^2 = -1$. Mostre que $\mathcal{Q}$ \'e um grupo com a multiplica\c{c}\~ao usual de matrizes.

\questao Seja $z  = a + bi \in \mathbb{C}$, onde $a$, $b \in \R$. Definimos $|z| = \sqrt{a^2 + b^2}$. Prove que $G=\{z \in \mathbb{C} \mid |z| = 1\}$ {\'e} um grupo
abeliano com a opera{\c c}{\~a}o de multiplica{\c c}{\~a}o de n{\'u}meros complexos.

\vesp

\questao Mostre que o conjunto $\Q[\sqrt{2}]^*=\{ a + b\sqrt{2} \in
\mathbb{R}^* \mid  a, b \in \Q \}$ {\'e} um grupo multiplicativo abeliano.

\vesp

\questao Mostre que os seguinte conjuntos s\~ao grupos, com as opera\c{c}\~oes indicadas:
\begin{enumerate}[label=({\alph*})]
  \item $(M_n(\R), +)$: o conjunto das matrizes $n \times n$ com entradas reais com a soma usual de matrizes. Este grupo \'e abeliano? (O resultado \'e verdadeiro se considerarmos matrizes sobre $\Q$ ou $\cmp$.)
  \item $(GL_n(\R), \cdot)$: o conjunto das matrizes invert{\'\i}veis $n \times n$ com entradas reais com o produto usual de matrizes. Este grupo \'e abeliano? (O resultado \'e verdadeiro se considerarmos matrizes sobre $\Q$ ou $\cmp$.)
\end{enumerate}

\vesp

\questao Seja $\Z_n = \Z/n\Z$ o conjunto dos inteiros m\'odulo $n$. Mostre que $(\Z_n, \oplus)$, onde para $\overline{a}$, $\overline{b} \in \Z_n$, $\overline{a} \oplus \overline{b} = \overline{a + b}$ \'e um grupo.

\vesp

\questao Seja $\Z_n^* = \Z_n-\{\overline{0}\}$. O conjunto $(\Z_n^*, \otimes)$, onde para $\overline{a}$, $\overline{b} \in \Z_n^*$, $\overline{a} \otimes \overline{b} = \overline{ab}$ \'e um grupo? Caso contr\'ario, qual a condi\c{c}\~ao sobre $n$ para que $(\Z_n^*, \otimes)$ seja grupo? 

\vesp

\questao Quais dos seguintes subconjuntos $G$ de $\Z_{13} = \Z/13\Z$ s{\~a}o grupos
com a opera{\c c}{\~a}o de multiplica{\c c}{\~a}o?
\begin{multicols}{2}
\begin{enumerate}[label=({\alph*})]
\item $G=\{\overline{1},\overline{12}\}$;

\item $G=\{\overline{1},\overline{5},\overline{8},\overline{12}\}$;

\item $G=\{\overline{1},\overline{2},\overline{3},\overline{4}, \overline{5},\overline{6},\overline{7},
 \overline{8},\overline{9},\overline{10},\overline{11},\overline{12}\}$
\item $G=\{\overline{1}, \overline{3},\overline{5},\overline{7},\overline{9},\overline{11}\}$.
\end{enumerate}
\end{multicols}

\vesp


\questao Seja $(G,*)$ um grupo com elemento neutro $e$.
\begin{enumerate}[label=({\alph*})]
\item Seja $G$ um grupo tendo $e$ como elemento neutro. Prove que se
$x^2=e$, para todo $x\in G$, ent{\~a}o $G$ {\'e} um grupo abeliano.
\item Mostre que se $x\in G$ {\'e} tal que $x^2=x$, ent{\~a}o $x$ {\'e} o elemento neutro.
\item Mostre que $(x^{-1})^{-1} = x$ para todo $x \in G$.
\item Mostre que $(a_1a_2\cdots a_{n - 1}a_n)^{-1} = a_n^{-1}a_{n - 1}^{-1}\cdots a_2^{-1}a_1^{-1}$, para todo $n \ge 2$.
\end{enumerate}

\vesp

\questao Sejam $(G, *)$ e $(H, \diamond)$ dois grupos. No conjunto $G\times H$ defina a opera\c{c}\~ao
\[
  (g_1,h_1)\cdot (g_2,h_2) = (g_1*g_2, h_1\diamond h_2).
\]
Mostre que $(G\times H, \cdot)$ \'e um grupo. Tal grupo \'e chamado de \textbf{produto direto} de $G$ com $H$.

\vesp

\questao Sejam $H$ e $K$ dois subgrupos de um grupo $G$. Mostre que $H \cap K$ \'e um subgrupo de $G$. De maneira geral, mostre que se $\{H_\alpha\}_{\alpha \in \Gamma}$ \'e uma fam{\'\i}lia de subgrupos de $G$, ent\~ao $\cap_{\alpha \in \Gamma}H_\alpha$ \'e um subgrupo de $G$.

\vesp

\questao Seja $G$ um grupo e seja $S$ um subconjunto de $G$. Mostre que $\ger{S}$ \'e o menor subgrupo de $G$ contendo $S$ e que $\ger{S}$ \'e a interse\c{c}\~ao de todos os subgrupos de $G$ que cont\'em $S$.

\vesp

\questao Sejam $H$ e $K$ subgrupos de um grupo $G$ (com nota{\c c}{\~a}o
multiplicativa).
\begin{enumerate}[label=({\alph*})]
\item Prove que $H\cup K$ {\'e} subgrupo de $G$ se, e somente se,
$H\subseteq K$ ou $K\subseteq H$.
\item Demonstre que $HK=\{hk \mid h\in H, k\in K\}$ {\'e} subgrupo
de $G$ se, e somente se, $HK=KH$.
\end{enumerate}

\vesp

\questao Seja $G$ um grupo abeliano e considere o subconjunto $T(G) = \{\alpha \in G \mid |\alpha| < \infty\}$. Mostre que $T(G)$ \'e um subgrupo de $G$. Tal subgrupo \'e chamado de \textbf{tor\c{c}\~ao} de $G$.

\vesp

\questao Seja $G = GL_2(\Q)$ e tome
\[
  A = \begin{bmatrix}
    0 & -1\\1 & 0
  \end{bmatrix},\quad B = \begin{bmatrix}
    0 & 1\\-1 & 1
  \end{bmatrix}.
\]
Mostre que $A^4 = B^6 = I$, onde $I$ \'e a matriz identidade, mas $(AB)^n \ne I$ para todo $n \ge 1$. Conclua que $AB$ pode ter ordem infinita apesar de $A$ e $B$ terem ordem finita (isso n\~ao acontece em grupos finitos).

\vesp

\questao Defina o \textbf{grupo especial linear} por
\[
  SL(2,\R) = \{A \in GL_2(\R) \mid \det(A) = 1\}.
\]
Prove que $SL(2,\R)$ \'e um subgrupo de $GL_2(\R)$.

\vesp

\questao Seja
\[
  \mathcal{V} = \left\{\begin{pmatrix}
    1 & 2 & 3 & 4\\
    1 & 2 & 3 & 4
  \end{pmatrix}, \begin{pmatrix}
    1 & 2 & 3 & 4\\
    2 & 1 & 4 & 3
  \end{pmatrix}, \begin{pmatrix}
    1 & 2 & 3 & 4\\
    3 & 4 & 1 & 2
  \end{pmatrix}, \begin{pmatrix}
    1 & 2 & 3 & 4\\
    4 & 3 & 2 & 1
  \end{pmatrix}\right\}.
\]
Verifique que $\mathcal{V}$ \'e um subgrupo de $S_4$.

\vesp

\questao Se $H$ e $K$ s\~ao subgrupos de um grupo $G$ tais que $|H|$ e $|K|$ s\~ao relativamente primos, ent\~ao $H\cap K = \{1\}$.

\vesp

\questao Seja $G$ um grupo de ordem 4. Prove que ou $G$ \'e c{\'\i}clico ou $x^2 = 1$ para todo $x \in G$. Conclua que $G$ \'e abeliano.

\vesp

\questao Seja $G$ um grupo e seja $g \in G$, $g \ne e$.
\begin{enumerate}[label=({\alph*})]
  \item Mostre que $g$ tem ordem 2 se, e somente se, $g = g^{-1}$.
  \item Mostre que se $|g| = mn$, ent\~ao $|g^m| = n$.
  \item Mostre que $|g^{-1}| = |g|$.
\end{enumerate}

\vesp

\questao Seja $G = S_3$, o grupo das permuta\c{c}\~oes de 3 elementos.
\begin{enumerate}[label=({\alph*})]
  \item Procure todos os subgrupos $H$ de $S_3$ com suas ordens.
  \item Para cada subgrupo $H$ de $S_3$, determinar as suas classes laterais \`a esquerda e \`a direita.
  \item Exiba um subgrupo pr\'oprio $H$ de $S_3$ tal que
  \[
    Hx = xH
  \]
  para todo $x \in S_3$.
  \item Exiba um subgrupo pr\'oprio $K$ de $S_3$ para o qual exista $x \in S_3$ tal que $Kx \ne xK$.
\end{enumerate}

\vesp

\questao Seja $H$ um conjunto n\~ao vazio de um grupo finito $G$. Mostre que $H$ \'e um subgrupo de $G$ se, e somente se, $H$ \'e fechado, isto \'e, se dados $a$ e $b \in H$, ent\~ao $ab \in H$. D\^e um exemplo que esta propriedade falha se $G$ for um grupo infinito.
\end{document}